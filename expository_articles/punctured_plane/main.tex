\documentclass[12pt]{article}
\usepackage[utf8]{inputenc}
\usepackage{graphicx}

\usepackage[english]{babel}
\usepackage[nottoc]{tocbibind}

\usepackage{hyperref}
\usepackage[left=3cm,right=3cm,top=2cm,bottom=2cm]{geometry}
%\setlength{\parindent}{0ex}

\usepackage{amsmath, amsfonts, amssymb, amsthm}
\newtheorem{thm}{Theorem}[section]
\newtheorem{lem}[thm]{Lemma}
\newtheorem{prop}[thm]{Proposition}
\newtheorem{cor}[thm]{Corollary}
\newtheorem{conj}[thm]{Conjecture}
\newtheorem{exmp}[thm]{Example}

\theoremstyle{definition}
\newtheorem{defn}{Definition}[section]

\title{Untangling loops in punctured $\mathbb{R}^2$}
\author{Sayantan Khan}
\date{July 2015}


\begin{document}

\maketitle

\section{The picture hanging puzzle}
Given two nails hammered into a wall, is it possible to hang a picture frame on it, given that you are allowed to wind the thread (and the thread's length not being a constraint) in any manner, so that removing any one nail makes the frame fall, but removing neither keeps the frame hanging. Even if one isn't familiar with algebraic topology, it's not too hard to figure out a solution. One solution that does work is the loop gotten by starting with a clockwise loop around nail 1, followed by a clockwise loop around frame 2, followed by a counter-clockwise loop around frame 1 and finally a counter-clockwise loop around frame 2.

This problem has a very natural generalization, namely the same problem but with $n$ nails, such that removing \emph{any} $k$ nails makes the picture fall, but removing any lesser number keeps the frame hanging. In this general setting, the way to approach the problem is not quite clear, but one can use algebraic topology to translate this problem into a purely algebraic problem, whose solution can be translated back into a solution of this problem.

\section{Translating the problem to algebraic topology}
We can look at a configuration of the thread around the nails as a path from $[0,1]$ to $\mathbb{R}^2$, which corresponds to the wall the nails are hammered on. Well, not quite. The important point here is that the thread cannot pass through the point a nail is hammered into. So the co-domain of the associated path will actually be $\mathbb{R}^2$ minus all the points the nails are hammered into, of which, there are $n$. When a nail is removed, it is possible that the thread can go through the point previously occupied by the nail, so the process of removing a nail changes the co-domain of the path.

The second point to note is that given some thread configuration, if the picture frame does fall, that means there is some continuous deformation of the path that takes it away from tall the nails, into a region that has no nails. In such a region, the path can be shrunk continuously to a point, so that must mean if the frame falls, the associated path can be continuously shrunk to a point. The converse is true, if one is willing to relax some of the physical conditions, namely friction. If the path can be shrunk, the weight of the frame combined with a lack of friction will cause the frame to fall.

The problem, in this setting really translates to finding a path such that it is cannot be shrunk to a point if there are $n$ holes in $\mathbb{R^2}$, but removing any $k$ of them makes it possible to shrink the path, but removing any fewer number doesn't work.

The paths that can be continuously deformed to one another correspond to the fundamental group of the space.

\section{Formalising the construction}
We'll begin by defining a few terms we'll use:

\begin{defn}
The topological space one gets by deleting $n$ distinct points from $\mathbb{R}^2$ is called an \emph{$n$-space}.
\end{defn}

\begin{lem}
The \emph{fundamental group} of an $n$-space is the free group of rank $n$.
\end{lem}

\begin{proof}
    It's easy enough to see that one can write an $n$-space as a union of open sets, each of which deformation retract to the circle, and the intersection of these spaces is path connected and non-empty. Seifert-van Kampen theorem tells us that the fundamental group of this space is the free product of $n$-copies of $\mathbb{Z}$, which is the free group of rank $n$.
\end{proof}

\begin{lem}[Nielsen-Schreier Theorem]\label{subg}
A subgroup of a free group is free.
\end{lem}

\begin{proof}[Sketch of proof]
    Consider a graph with with one vertex $v$ and $n$ edges $e_1, e_2, \ldots , e_n$ from itself to itself. The fundamental group of this graph is the free group of rank $n$ by the previous lemma. Any subgraph of the fundamental group will correspond to a cover of the graph whose fundamental group is isomorphic to the given subgroups.  Every covering space of a graph is a graph. The fundamental group of a graph is free. This shows that a subgroup of a free group is free.
\end{proof}

\begin{lem}\label{impo}
Two elements $a$ and $b$ of a free group commute with each other iff both $a$ and $b$ are perfect powers of some element $x$.
\end{lem}

\begin{proof}
If $a$ and $b$ are perfect powers of some element $x$, it's obvious that they commute; it follows from the fact that group multiplication is associative. The converse will use the Nielsen-Schreier theorem. Assume that $a$ and $b$ are not perfect powers of some word $x$ in the free group. That means $a^m \neq b^n$ for all pairs of integers $(m,n)$ except $(0,0)$. Consider the subgroup $H$ generated by $a$ and $b$. Since $a$ and $b$ commute, any element of $H$ can be uniquely written as $a^{c_a}b^{c_b}$, where $c_a$ and $c_b$ are integers. This means that the subgroup $H$ is isomorphic to the group $\mathbb{Z}^2$. But this means we have a subgroup of a free group which is not free, but that is false by lemma \autoref{subg}. Hence, our assumption that $a$ and $b$ are not perfect powers of some word $x$ must be false. Hence, if $a$ and $b$ commute, they must be of the form $x^n$ and $x^m$ for some word $x$ and some integers $m$ and $n$.
\end{proof}

\begin{defn}
The commutator of a list of elements is analogous to the commutator of two elements, i.e. $[a_1, a_2, \ldots, a_n]$ is defined as $a_1a_2\ldots a_na_1^{-1}a_2^{-2}\ldots a_n^{-1}$.
\end{defn}

\begin{lem}
The commutator of two elements is identity iff they commute.
\end{lem}

\begin{proof}
If they commute, then the commutator $aba^{-1}b^{-1}$ reduces to identity. If $aba^{-1}b^{-1}=e$, then $ab=ba$, which means they commute. 
\end{proof}

\begin{lem}\label{impo2}
In a free group, if the word $[a_1, a_2, \ldots, a_m]$ ($m>1$) is $x^n$ for some $x$ in the group and some positive integer $n$ and the $a_i$s are members of the basis, then $n=1$, and $x$ is the word itself.
\end{lem}

\begin{proof}
If $x$ was cyclically reduced, so would $x^n$, and if $x$ were not cyclically reduced, neither would $x^n$. This means that in this case, $x$ is cyclically reduced. Powers of cyclically reduced words are obtained by just formal concatenation. Which means $x^n$ is just $x$ concatenated $n$ times, and hence every letter of $x$ is repeated $n$ times in $x^n$. But each letter in the word occurs exactly once, which means $n=1$.
\end{proof}

\begin{thm}
For any $n$-space, there exists a loop such that filling in any $k$ of the $n$ holes makes it \emph{nullhomotopic}, but filling in any lesser number of holes leaves it non trivial.
\end{thm}

\begin{proof}
The fundamental group of an $n$-space is the free group on $n$ generators. So any loop in the space is an element of the fundamental group and filling in $k$ holes corresponds to the homomorphism that maps $k$ basis elements to identity and the rest to non trivial elements. To be more precise, if the holes $\{h_i\}$ are filled up, it corresponds to the homomorphism which maps the $a_i$ basis element to identity if the hole $h_i$ is filled up, otherwise, it maps the basis element to itself. Our goal is to find the an element in the free group of rank $n$ such that the homomorphism corresponding to filling up $k$ or more holes maps it to the identity, and any homomorphism corresponding to filling up less than $k$ holes does not map it to the identity.
If $k=n$, the answer is trivial, just take the word $abc\ldots n$. This word satisfies the required properties. For $k<n$, consider the sequence of words, $\{c_i\}$, where $c_i$ is the commutator of the $i$th collection of $(k+1)$ letters from the $n$ basis letters. There clearly are $n \choose {k+1}$ such commutators. The word $W$ as described below satisfies the properties.
$$W = \left[ \ldots [[[a^{-1}b^{-1}c^{-1}\ldots n^{-1}, c_1], c_2], c_3] \ldots c_{n \choose{k+1}}\right]$$
Here's why $W$ satisfies the required properties. If at least $k$ basis elements get mapped to the identity, at least one of the $n \choose{k+1}$ commutators will map to identity, which means the whole compound commutator will collapse to identity. This shows that the first property is satisfied.

To show that $W$ satisfies the second property, one first needs to show that none of the simple commutators $c_i$ map to identity under the homomorphism. This is obvious because the homomorphism sends at most $(k-1)$ basis elements to identity, and the simple commutators have $(k+1)$ elements, which means their image will have at least $2$ elements, which does not evaluate to identity. Now one needs to show that the compound commutator does not collapse either. To prove that, start from the deepest nested commutator, which is $[a^{-1}b^{-1}c^{-1}\ldots n^{-1}, c_1]$. The image of the this commutator under the homomorphism will leave at least $(n-k+1)$ terms in the left hand term. From lemma \autoref{impo}, we know for this commutator to collapse, they elements have to be perfect powers of some element in the free group, and from lemma \autoref{impo2}, we know that element has to be $c_1$. This means the left hand term will have to be a perfect power of $c_1$ for the commutator to collapse. But that's not possible because the first letter in the left hand term has a negative exponent whereas the first letter in any power of $c_1$ will have a positive exponent. Extending this argument, and reducing the nesting level, we can say the same for the commutator with $c_2$, $c_3$ and so on. This shows the word does not collapse if the homomorphism maps at most $(k-1)$ basis elements to identity.

This concludes the proof and shows the existence of such a loop by explicit construction.
\end{proof}

The element of the group one gets from the previous theorem is precisely the configuration of thread we want that will satisfy our required constraint.
\end{document}
